\chapter{Methodology}
\label{chap:methodology}
The main objective of multivariate survival modelling is to understand and quantify factors that influence the time until an event occurs. In this study, the event of interest is the time to onset of a first chronic disease. Survival analysis entails the estimation of a \textit{survival function}, denotes as \begin{math}
    S(t)
\end{math}, which represents the percentage of participants for which the event has not occurred up to and including time \textit{t}. A survival curve may be interpreted as the individual probability of living past time \textit{t} without onset of a first chronic disease. Consequently, a survival curve is defined as a collection of \textit{hazard rates h(t)}, where \textit{h(t)} is the probability of the event occuring at time \textit{t} given that it has not occured up to time \textit{t-1}. The effect of covariates on time-to-event can be assessed by comparing survival curves for different covariates. 

This section provides a comprehensive overview of the \textit{Cox Proportional Hazard}(CPH) model, which is widely recognized as the most commonly used model in multivariate time-to-event analysis. In addition, this section highlights several extensions and specifications of the CPH model, such as the inclusion of time-dependent covariates and the choice of time-scale. 
\medskip
\section{Cox Proportional Hazard Model}
\label{section:data:CPH}
The CPH model is a \textit{regression} model that attempts to model the hazard rate $h(t|x)$ as a function of time $t$ and covariates $x$. Mathematically, the CPH is represented by: 

$$
\underbrace{h(t \mid x)}_{\text {hazard }} = \overbrace{b_0(t)}^{\text {baseline hazard }} \underbrace{\exp \overbrace{\left(\sum_{i=1}^n b_i\left(x_i-\overline{x_i}\right)\right)}^{\text {log-partial hazard }}}_{\text {partial hazard }}
$$

According to this specification, the log-hazard of an individual is a linear function of their covariates $x$ and a population-based baseline hazard $b_{0}(t)$. Note that the only time-component in this model is the baseline hazard. The partial hazard, which is dependent on the subject specific covariates, is the time-invariant scaling factor that either inflates or deflates the baseline hazard. This also implies that survival curves can never cross each other. The baseline hazard can be estimated using different methods, such as \textit{Breslow}. %info

A fundamental assumption of the standard CPH is that the hazard ratio adheres to the \textit{proportional hazard assumption}. This assumption implies that the hazard ratio is constant over time for all levels of the covariates. The proportional hazard assumption should be tested and handled if violated. Violation results in biased and unreliable results, and can lead to misinterpretation of factors that influence survival. There are several approaches which address violation of the proportional hazard assumption, such as stratification or the use of time-dependent covariates. In stratified proportional hazard models, seperate Cox models are fit on different groups, which allows these groups to have a different baseline hazard. Another approach is to allow for time-varying covariates in a Cox model. This model, hereafter referred to as the \textit{Extended Cox Model} (ECM), allows the hazard ratios to vary over time and provides a more accurate assessment of the impact of time-varying covariates on the event of interest.  This is the model that will be considered in this paper. 
\section{Extended Cox Model (Cox's Time-varying Model)}
\label{section:data:ecm}
The CPH model can be extended in such a way that it can incorporate covariates $x_{i}(t)$ that change over time. This characteristic is of great importance to clinical follow-up studies, as datasets usually include both baseline (time-independent) and time-dependent covariates. Mathematically, the ECM is represented by:  

$$
h(t \mid x(t))=\overbrace{b_0(t)}^{\text {baseline }} \underbrace{\exp \overbrace{\left(\sum_{i=1}^n \beta_i\left(x_i(t)-\overline{x_i}\right)\right)}^{\text {log-partial hazard }}}_{\text {partial hazard }}
$$

The package that is most commonly used for survival analysis in R is \textit{Survival}. To use this package, the dataset must be organized in a \textit{long} format, where each individual is represented by a unique \textit{project\_pseudo\_id} variable and each time interval is identified by \textit{start} and \textit{stop} variables. 
The interval is considered to be open on the left and closed on the right, implying that any variable linked with an interval is presumed to be measured at the \textit{stop} time. Furthermore, both baseline and follow-up covariates are included, and a time-dependent binary \textit{event} variable which indicates whether or not an interval ends in the occurence of the event of interest. 
% wat moet hier nog meer bij? Iets over estimation methods for regression parameters
% incidence enzo moet bij data sectie denk ik? 
However, prediction of survival curves and time-to-event is not trivial when dealing with time-varying covariates, because this would require knowledge on the covariate values beyond the observed times.  %Bridge to dynamic survival modelling 


Time-varying covariate values would be taken as the values at the start of each interval.

\section{Censoring and choice of time-scale}
\label{section:data:timescale_censoring}
% Cencoring and choice of time-scale
% Meer op cencoring? 
\noindent Inherent to making inferences about duration is the presence of censoring and the choice of time-scale in the ECM. Censoring is especially common in clinical studies where event status updates and covariates are collected during periodic follow-up assessments. Generally, there are three variations; left-, right- and interval-censoring. Participants who have not developed a chronic disease before the end of the follow-up period are labeled as right-censored. Right censoring can either occur due to end-of-study or due to loss-to-follow-up. When a participant enters the study with a disease of interest already present, this is a case of left-censoring. Interval-censoring occurs when the event occurs inbetween two clinical assessments, and the exact time of incidence is unknown. 

The Lifelines dataset \ref{chap:data} has three main assessments. Chronic disease presence and incidence is self-reported in questionnaires. For each condition, presence is reported at the baseline assessment 1a, and incidence is reported at follow-up assessments 2a and 3a. The question on disease follow-up is formulated as 'did the health problems listed below start since the last time you filled in the lifelines questionnaire?'. This question will inherently result in interval-censoring, because it does not provide the specific incidence time of disease. 

There are several techniques on how to fit a ECM with interval-censored data. %HOE FIX JE DAT
%https://onlinelibrary.wiley.com/doi/full/10.1002/sim.9645


Left-censored individuals are not taken into account, because it is impossible to evaluate the association of time-varying covariates with chronic disease incidence when the event has already occured.  



Although survival analysis is designed to deal with estimation of right-censored data, it is important to understand the risks of underestimating the true survival time when ignoring right-censored individuals. 

Another challenge inherent to duration is the choice of time-scale, especially in the analysis of large-scale health surveys where the interest is to find the association of risk factors with the development of a disease. In this case, either the time since the baseline survey (time-on-study), or age can be used as time-scale. %Papers on this 
Kom et al. (1997) \cite{kom1997time} propose that age, with stratification for birth cohort effects, is the appropriate time scale in Cox proportional hazard regression models that analyze health data from longitudinal studies.  
