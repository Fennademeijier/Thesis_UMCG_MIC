\chapter{Introduction}\label{chap:introduction}
%\addcontentsline{toc}{chapter}{Introduction}
% - Highlighting the relevance
Ageing is defined as the process of time-dependent functional decline of biological organisms and manifests itself on a variety of physiological scales \cite{lopez2013hallmarks}. It is the result of accumulation of cellular and molecular damage over time, and is the highest risk factor for susceptibility to diseases and ultimately death. The progression of ageing relies on the balance between exposure and resilience to damaging factors, which are both subject to the heterogeneity of individuals. Huge differences in human lifespan suggest that there exist underlying differences in ageing processes and that the passage of time is not the ideal measure of ageing speed \cite{sprott2010biomarkers}. Consequently, a person's chronological age is merely a proxy for the rate of ageing, but not a reliable reflection of general health status. Instead, ageing can be quantified through the identification and measurement of biomarkers in the form of biological age. Predicting and identifying factors that influence biological age has rapidly gained popularity in the field of ageing research over the years. Given that human life expectancy is continuously increasing, but not in parallel with increased healthy lifespan, understanding the biological cause of ageing and increasing health in elderly has become one of the most urgent societal and scientific challenges of today.

% History of biomarker research --> data emergence --> AI application
Previous studies on the progression of ageing have deployed diverse modelling techniques that aim to capture the impact of health conditions on age. The more simple approaches include integrating multiple variables into a low-dimensional representation of age, such as a frailty index defined by the proportion of accumulated health deficits \cite{mitnitski2001accumulation}. Alternatively, multivariate linear regression techniques have been explored to develop formulas \cite{levine2013modeling} and prediction models \cite{bae2008development} for biological age, as well as a DNA methylation (DNAm) ageing clock \cite{horvath2013dna}. Other papers employ principle component analysis (PCA) to assess biological age, in which the 1st principle component score is transformed into biological age using a T-score transformation \cite{nakamura1988assessment},\cite{park2009developing}. However, given increased computational power and availability of large medical datasets, there is a potential to advance our understanding of the complex and high-dimensional ageing process even more \cite{farrell2021potential}. This observation fuelled the emergence of advanced deep learning based approaches, such as a deep hematological aging clock  and a deep learning prediction model of biological age based on electronic medical records \cite{wang2017predicting}. 

Nevertheless, there is a lot of ground to be gained as 
Such techniques take chronological age as the dependent variable and label the predicted variable as the biological age. By design, all of such supervised biological age prediction techniques aim to minimize the difference between chronological age and predicted biological age of an individual. Approach suffers from a theoretical contradiction. 



ith advent algorithms that employ deep learning, through non-linear transformations, it is now possible to better handle data incompleteness, inaccuracy, and scalability to learn from EMR. For instance, DeepPatient uses denoising Autoencoder (dAE) t

-- deep ageing clocks

However, there is a lot of ground to be gained as 

--> complexity non-linear process

The merging of multiple measures into a single biomarker of may prove useful in both biological research - to study how lifestyle, environment and evolution impact ageing speed - as well as in public health research or clinical practice - to indentify individuals at increased risk of disease. 

Advanced analytical methodologies in pattern recognition and computational learning, as Machine Learning approaches, can also be employed to explore factors associated with the metric of health.
Even so, the aging process is complex and has multiple interacting physiological scales from the molecular to cellular to whole tissues. In the face of this complexity, we can significantly advance our understanding of aging with the use of computational models that simulate realistic individual trajectories of health as well as mortality. 
%voorbeelden toepassing AI op health data. 
% - Highlighting some previous research 
    %- Emergence of AI in ageing research
    %- Examples of AI in ageing research, biomarkers of ageing --> we have to better understand the observed correlations between the




% - Toegevoegde waarde van deze research
    % - INtroducing the data and methodology

with artificial intelligence methods advancing and large data sets becoming more publicly available, there is an opportunity to deepen the understanding of multiple underlying mechanisms that influence the rate at which people age. 



% PATRICK GROENEN
%- Als lezer niet voorbereidt op de research question : Introductie! 
%- Wat is het nou eigenlijk wat ik wil onderzoeken? 
%Ui-model. Introductie, grote stappen snel thuis. 1e zin is hele wereld. In introductie bouw je op naar problemen die er zijn, Spanning bouwt op. Dan komt mijn moment of fame. Logisch dat je dat wil gaan oplossen. Mooie manier. 
% Introducing data and methodology, target of model
